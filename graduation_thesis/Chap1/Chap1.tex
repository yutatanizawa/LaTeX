%!TEX encoding = UTF-8 Unicode

\section{緒論}
\subsection{背景}
近年,高性能計算 (High Performance Computing,HPC)システムの用途は多様化し,専門知識を持たない利用者が容易にHPCシステムを利用する需要が高まっている.一般的に,コマンド操作に基づいてHPCシステムを操作する利用環境や利用するHPCシステムごとに異なる操作方法により,HPCを専門としない研究者はHPCシステムを使いこなすために多くの学習時間を費やす必要がある.実際にPingらは,学習目的でHPCシステムを初めて利用する学生などはHPCシステム利用環境の構築に多くの時間を費やしてしまい,本来の目的である学問のためのHPCシステムの利用を達成するまでに多大な時間を費やしてしまうという問題を指摘している\cite{cite1}.そこで,従来のコマンド操作に基づく利用環境や,システムごとに異なる利用方法を利用者から隠蔽し,ウェブブラウザを用いて容易かつ統一的にHPCシステムを利用することが可能なウェブインタフェースの研究開発が行われている\cite{OOD_1}.\par
現在用いられているジョブスケジューラには数多くの種類が存在し,今後も多くのジョブスケジューラが開発されることが予想される.そのため,ウェブインタフェースはより多くのジョブスケジューラに対応することが求められる.しかし,既存のウェブインタフェースが新たなジョブスケジューラへの対応をするたびに,開発者はウェブインタフェース本体を改修する必要がある.そのため,ウェブインタフェースの保守性に問題があるといえる.\par

\subsection{目的}
本研究では,スーパーコンピュータ利用環境において,ジョブスケジューラの多様化に伴い発生し得るウェブインタフェースの保守性の問題に着目する.そこで,既存のウェブインタフェースの機能を以下の2つの機能に分離する.1つは,ユーザがウェブブラウザ上でHPCシステム利用を可能とする機能 (ウェブ機能).もう1つは,多様なジョブスケジューラを統一的に取り扱う機能 (スケジューラ抽象化機能).既存のウェブインタフェースをウェブ機能とスケジューラ抽象化機能に分離することで,新たなジョブスケジューラに対応させる際のウェブインタフェースの改修箇所はスケジューラ抽象化機能のみとなる.そのため,提案手法を用いて,ウェブインタフェースの保守性の問題を解決することを目的とする.具体的には,既存のウェブインタフェースの機能を分離し,実際のシステムに実装する.実装における動作の確認を行い,提案手法の実現可能性を示す.さらに,既存のウェブインタフェースと提案したウェブインタフェースのジョブ投入時のオーバヘッドを定量的に評価することで提案手法の有用性を示す.\par

\subsection{本論文の構成}
本論文は全5章から構成される.第1章では,本研究の背景と目的について述べた.第2章では,関連研究について説明し,既存のウェブインタフェースについて述べる.第3章では,ウェブインタフェースを介したHPCシステム利用環境について説明し,提案手法の実装と動作の確認を行う.第4章では,実装した提案手法の評価結果を示し,その考察を行う.第5章では,本研究の結論と今後の課題を述べる.\par
