%!TEX encoding = UTF-8 Unicode

\section{緒論}
\subsection{背景}
近年,高性能計算(High Performance Computing,HPC)システムの用途は多様化し,専門知識を持たない利用者が容易にHPCシステムを利用する需要が高まっている.一般的に,コマンド操作に基づいてHPCシステムを操作する利用環境や利用するHPCシステムごとに異なる操作方法により,HPCを専門としない研究者はHPCシステムを使いこなすために多くの学習時間を費やす必要がある.実際にPingらによると,学問のためにHPCシステムを初めて利用する学生などはHPCシステムのための利用環境の構築に多くの時間を費やしてしまい,本来の目的である学問のためのHPCシステムの利用までの多大な時間を費やしてしまうという問題が挙げられている.\cite{cite1}そこで,従来のコマンド操作に基づく利用環境や,システムごとに異なる利用方法を利用者から隠蔽し,ウェブブラウザを用いて容易かつ統一的にHPCシステムを利用することが可能なウェブインタフェースの研究開発が行われている.\par
しかし,現行のウェブインタフェースはユーザへの簡易かつ統一的なHPC利用環境とを提供するため,ウェブインタフェースの機能の改修を行う度にウェブインタフェース本体を改修する必要がある.そのためシステムの保守性に問題があるといえる.また,HPCシステムの利用者にとって,様々なジョブスケジューラを統一的に扱うということは○○や○○など,様々な分野において有用な研究であり,関心が高い分野である.\par
そのため,ウェブインタフェースの機能をユーザがウェブブラウザ上でHPC利用を可能とする機能と多様なジョブスケジューラを統一的に取り扱う機能の二つの機能に分離して実装することで前述した課題点を解決し,ジョブスケジューラ統一化の要望にも応用できる機能の分離利用が可能なウェブインタフェースを考えることができる.\par

\subsection{目的}
本研究では,HPCシステムの利用難度の高さやHPCシステムにおけるジョブスケジューラの多様化に伴い発生し得るHPCシステム利用環境に関する課題点に着目する.インタフェースをウェブ機能とスケジューラ抽象化機能に分離することで操作の簡易化や保守性などの問題を解決することを目的とする.具体的には現行のインタフェースの機能を分離し,実装を行う.実装における動作の確認を行い,提案したインタフェースを定量的に評価することで提案手法の有用性を示す.\par

\subsection{本論文の構成}
本論文は全5章から構成される.第1章では,本研究の背景と目的について述べた.第2章では,関連研究について説明する.第3章では,ウェブインタフェースを介したHPCシステム利用環境について説明し,提案手法の実装を行う.第4章では,実装の評価結果を示し,その考察を行う.第5章では,本研究の結論と今後の課題を述べる.\par
