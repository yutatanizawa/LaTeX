%!TEX encoding = UTF-8 Unicode
\section{結論}
近年,HPCシステムの用途は多様化し,専門知識を持たない利用者がHPCシステムを利用する需要が高まっている.HPCの利用にはコマンド操作に基づいた利用環境や,システムごとに異なる操作方法などHPCシステムを利用するためには多くの学習時間が必要とされている.このような課題点を解決するためにウェブブラウザを用いて容易かつ統一的にHPCシステムを利用することが可能なウェブインタフェースについての研究開発が多く行われている.しかし,既存のインタフェースは新たなジョブスケジューラへの対応を行うたびに,ウェブインタフェース本体を改修する必要があるため,保守性に大きな問題を抱えている.\par
本研究では,多様なジョブスケジューラへの対応に伴い発生し得るウェブインタフェースの保守性の問題に着目し,HPCシステムの簡易な利用環境とウェブインタフェースの保守性を併せ持つ実装について考えた.この実現のために,ウェブ機能とスケジューラ抽象化機能を分離し,それぞれ独立に保守管理する手法を提案した.\par
第2章では,本研究の関連研究について説明した.はじめに,HPCシステム利用環境について説明した.その後,ウェブインタフェースであるOODについて説明し,OODの機能と設計について説明した.さらに,既存のインタフェースの課題点について説明し,本研究で着目すべき点を明らかにした.\par
第3章では,ウェブインタフェースを介したHPCシステム利用環境として提案手法の説明と実装を行った.はじめに,従来手法と提案手法の説明を行った.従来手法と提案手法を比較して,提案手法の利点や設計の変更点について説明した.その後,提案手法の実装を行った.はじめに,実装の概要について説明した.その後,スケジューラ抽象化機能とNQSVの連携を行った.スケジューラ抽象化機能自身がジョブの状態を管理する設計を行ったことにより,スケジューラ抽象化機能の汎用性をより高めることができたといえる.続いて,ウェブ機能とスケジューラ抽象化機能との連携を行った.ウェブ機能がスケジューラ抽象化機能を呼び出すことで,ウェブ機能としてOODを用い,スケジューラ抽象化機能であるPSI/Jを経由して,指定したスケジューラにジョブの投入や削除を行うことができるようになった.\par
第4章では,機能分離に伴いオーバヘッドの発生が懸念されるため,実装の評価を行った.はじめに,評価環境,評価条件について説明した.その後,ジョブ実行時のオーバヘッドの評価を行った.提案手法で発生が懸念されたオーバヘッドは機能分離前のターンアラウンドタイムの約5%以下であることがわかり,提案手法の実装によって生じたオーバヘッドは充分に小さいということが明らかになった.\par
以上の結果から,HPCシステム利用環境におけるウェブインタフェースをウェブ機能とスケジューラ抽象化機能の二つに分離し,それぞれ独立に保守管理することの実現可能性と有用性をしめすことができた.\par
今後の課題として,PSI/Jへのジョブの一旦停止 (hold)機能と再開 (release)機能の実装が挙げられる.OODではhold機能とrelease機能が実装されていることに対して,PSI/Jではそれらの機能に対応していない.そのため,これらの操作に必要な両者の連携を設計して実装していくことで,より多様なジョブスケジューラやその使い方に対応可能となると期待される.

