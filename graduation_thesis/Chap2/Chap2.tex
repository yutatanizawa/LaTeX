%!TEX encoding = UTF-8 Unicode

\section{関連研究}

\subsection{緒言}
本章では関連研究のついて述べる.初めに一般的なHPCシステムの利用方法,続いてOpenOnDemandと呼ばれるインタフェース,最後に現行のインタフェースにおけるの課題を述べる.

\subsection{HPCシステム利用方法}
HPCシステムとは,スーパコンピュータやコンピュータクラスタの能力を利用して,ほかのコンピュータを遥かに凌ぐ速度で計算課題(ジョブ)を処理し,実行するシステムを指す.このようなコンピューティング能力の集約によって,さまざまな科学分野において他の方法では対処できない大きな課題を解決できる.実際に,平均的なデスクトップコンピューターは毎秒数十億の計算を実行できる.これは,人間が複雑な計算を行うことができるスピードに比べれば,素晴らしい数字である.しかし,HPCシステムは,1秒に数千兆の計算を実行することができるため,大規模な課題に対してはより適しているといえる.\par
一般的にHPCシステムは数種類のサーバにより構成される.ジョブを実行するためのワーカーノード,ワーカーノードを管理するためのマスターノード,ユーザ情報の管理を行うログインサーバ,実行するジョブのファイルなどを保存管理するファイルサーバなどである.ユーザは自身のPCからこれらの複数のサーバが連携された

\subsection{OpenOndemand}

\subsection{現行のウェブインタフェースにおける課題}

\subsection{結言}
