%!TEX encoding = UTF-8 Unicode

\section{関連研究}

\subsection{緒言}
本章では関連研究のついて述べる.初めに一般的なHPCシステムの利用方法,続いてOpenOnDemandと呼ばれるインタフェース,最後に現行のインタフェースにおけるの課題を述べる.

\subsection{HPCシステム利用方法}
HPCシステムとは,スーパコンピュータやコンピュータクラスタの能力を利用して,ほかのコンピュータを遥かに凌ぐ速度で計算課題(ジョブ)を処理し,実行するシステムを指す.このようなコンピューティング能力の集約によって,さまざまな科学分野において他の方法では対処できない大きな課題を解決できる.実際に,平均的なデスクトップコンピューターは毎秒数十億の計算を実行できる.これは,人間が複雑な計算を行うことができるスピードに比べれば,素晴らしい数字である.しかし,HPCシステムは,1秒に数千兆の計算を実行することができるため,大規模な課題に対してはより適しているといえる.\par
一般的にHPCシステムは数種類のサーバにより構成される.ジョブを実行するためのワーカーノード,ワーカーノードを管理するためのジョブスケジューラが搭載されたマスターノード,ユーザ情報の管理を行うログインサーバ,実行するジョブのファイルなどを保存管理するファイルサーバなどである.ユーザはログインノードに格納されているユーザ情報を用いてマスターノードにログインする.ユーザはファイルサーバからジョブを参照して,マスターノードに実行を依頼する.\par
一方で,ユーザは利用したいクラスタを遠隔で操作するために自身のコンピュータから鍵の登録を行った後,SSH接続を用いてクラスタに接続する.そして,ユーザは利用するジョブスケジューラの種類に応じた形式でジョブスクリプトを作成する.その後,与えられたコマンドを用いてマスターノードはジョブキューにジョブを投入する.マスターノードのジョブスケジューラが実行するジョブの管理を行い,実行が終了したジョブは標準出力とエラーファイルが出力され,ジョブの実行結果を確認することができる.\par

\subsection{Open Ondemand}

\subsection{現行のウェブインタフェースにおける課題}

\subsection{結言}
